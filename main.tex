\documentclass[12pt]{article}
\usepackage{graphicx}
\usepackage[a4paper, margin=2.5cm]{geometry}
\usepackage{float}
\usepackage{tocloft}
\usepackage[brazilian]{babel}

\addto\captionsbrazilian{%
  \renewcommand{\contentsname}{Sumário}%
  \renewcommand{\figurename}{Figura}%
}

\title{\fontsize{24}{28}\selectfont Redes Neurais}
\author{\fontsize{14}{16}\selectfont JP \\ PL \\ DB}
\date{}

\begin{document}

\begin{figure}[t]
  \centering
  \includegraphics[width=0.5\textwidth]{logo-fatec.png}
\end{figure}
\begin{titlepage}
    \centering
    \vspace*{4cm}
    {\LARGE\bfseries \fontsize{24}{28}\selectfont Redes Neurais - Relatório\par}
    \vspace{0.25cm}
    {\Large \fontsize{18}{22}\selectfont Classificação de flores\par}
    \vfill 
    \centering
    \begin{tabular}{c}
        \fontsize{14}{16}\selectfont Diego Brunetto da Silva \\
        \fontsize{14}{16}\selectfont João Pedro Borges Baeta \\
        \fontsize{14}{16}\selectfont Pedro Henrique Levy Fermino Ferreira
    \end{tabular}
    \vspace*{\fill}
    
    \centering
    \begin{tabular}{l}
        \textbf{Professor} \\
        Alexandre Garcia de Oliveira
    \end{tabular}
\end{titlepage}

\tableofcontents
\newpage

\section{O conceito de Redes Neurais}

Uma rede neural é um sistema de computador com nós interconectados que funcionam como os neurônios do cérebro humano. Usando algoritmos, eles podem reconhecer padrões ocultos e correlações em dados brutos, agrupá-los e categorizá-los, e aprender e melhorar continuamente com o tempo. Redes neurais artificiais (ANNs) consistem em uma camada de nós com uma camada de entrada, uma ou mais camadas ocultas e uma camada de saída. Cada nó ou neurônio artificial é conectado a outro e possui peso e limite apropriados. As redes neurais dependem de dados de treinamento para aprender e melhorar sua precisão ao longo do tempo.

\section {Peso e Bias}

Nas redes neurais, entradas parciais são dadas aos neurônios artificiais. Cada entrada é ponderada. Os pesos representam o desempenho de uma entrada específica. Quanto maior o peso de uma entrada, maior sua influência na rede neural. O viés, por outro lado, é como uma interceptação adicionada a uma equação linear. É um parâmetro adicional em uma rede neural que ajusta a saída do neurônio, bem como a soma ponderada de suas entradas. Simplificando, Bias é uma entrada com valor "1" associado ao peso "b" de cada neurônio. Sua função é aumentar ou diminuir a entrada líquida para transladar a função de ativação no eixo.

\section{Classificação de flores}

Escolhemos realizar a classificação de duas flores (rosas e girassóis) por meio da implementação de um algoritmo de redes neurais, utilizando padrões naturais de ambas as flores para prever qual é a flor apenas com a inserção desses parâmetros.

\section{As flores a serem classificadas}

Como mencionado anteriormente, as duas flores escolhidas para realização das classificações foram as rosas e os girassóis. Adotamos os dois principais padrões naturais destas flores, que são:\\

Rosas
\begin{itemize}
    \item Tamanho - Variam entre 5 e 10cm de diâmetro;
    \item Cor - A característica mais predominante de uma rosa, adotamos o número 0 para essa variável;
    \item Rótulo (Label) - Adotamos o número 0 para rosa, para identificação da rede neural.
\end{itemize}

Girassóis
\begin{itemize}
    \item Tamanho - Variam entre 20 e 30cm de diâmetro;
    \item Cor - Os girassóis não possuem como característica mais importante sua cor, portanto adotamos o número 1 para essa variável;
    \item Rótulo (Label) - Adotamos o número 1 para girassol, para identificação da rede neural.
\end{itemize}

\begin{figure}[H]
  \centering
  \includegraphics[width=0.5\textwidth]{o-girassol-e-a-rosa-14280197-140920181943.jpg}
  \caption{Rosa e girassol}
\end{figure}

\section{Função do Erro}

\begin{figure}[H]
  \centering
  \includegraphics[width=1.1\textwidth]{funcao_erro.png}
  \caption{Função Erro}
\end{figure}

No código a função que calcula o erro foi inserida dentro do loop que treina a rede neural por meio da diferença entre o rótulo (label) esperado e a predição, sendo armazenado na variável erro\_total para cada padrão de flor encontrado no loop.

\begin{figure}[H]
  \centering
  \includegraphics[width=0.8\textwidth]{funcao_erro2.png}
  \caption{Média Função Erro}
\end{figure}

Ao final de cada época, o loop divide o erro total pelo número de padrões de flor, assim obtendo o erro médio da época, atualizando os pesos da rede neural para refinar o treinamento.

\section{Sigmóide}
\begin{figure}[H]
  \centering
  \includegraphics[width=0.8\textwidth]{sigmoide.png}
  \caption{Função de ativação sigmóide e derivada da função sigmoide}
\end{figure}

\section{Hiperparâmetros}
\begin{figure}[H]
  \centering
  \includegraphics[width=0.5\textwidth]{hiperparametros.png}
  \caption{Taxa de aprendizado e épocas}
\end{figure}

\textbf{Taxa de aprendizado}\\
A taxa de aprendizado (learning rate) é um hiperparâmetro que determina o ajuste dos pesos dos neurônios de um algoritmo de rede neural.\\
Esse hiperparâmetro deve ser ajustado de maneira correta, pois um valor ou muito alto ou muito baixo pode comprometer o desempenho e a acurácia da predição a ser realizada pelo algorítmo.\\
O valor 0.1 de taxa de aprendizado foi adotado por ser considerado um valor moderado e comumente utilizado em algoritmos de redes neurais. Isso significa que: a cada atualização nos pesos, a taxa de aprendizado vai ajustar cada um dos pesos em 10\%.\\
\\
\textbf{Épocas}\\
O hiperparâmetro épocas (epochs) é o número de vezes que a rede neural irá testar os padrões apresentados.\\
Foi adotado o valor de 1500 épocas no código apresentado, o que significa que os padrões apresentados serão lidos 1500 vezes, mudando os pesos conforme o erro calculado em cada amostra realizada no treinamento da rede neural.\\
O número ideal de épocas visa velocidade e confiabilidade, tentando minimizar os erros e atingir o valor erro mais próximo de zero.\\
\textit{Caso tenha interesse, procure pelos gráficos do gradiente descendente (Item 9, Resultados) com 1500, 500 e 100 épocas, e compare com o erro médio.}

\section{Gradientes}

\begin{figure}[H]
  \centering
  \includegraphics[width=1\textwidth]{gradiente.png}
  \caption{Calculo dos gradientes}
\end{figure}

A função da rede neural é dada por:

\begin{equation}
w_1 \cdot \text{cor} + w_2 \cdot \text{diâmetro} + b \rightarrow \textit{chamaremos de weighted\_sum}
\end{equation}\\
O output que é a classificação das flores será a aplicação do sigmoid a função weighted\_sum, após isso será possível encontrar o vetor gradiente para calcular as derivadas parciais: 

\begin{equation}
\text{output} = \left(1 + \text{sigmoide}\left(-\left(w_1 \cdot \text{cor} + w_2 \cdot \text{diâmetro} + b\right)\right)\right)
\end{equation}\\
O vetor gradiente será dado pela derivada parcial de w1, w2 e b, abaixo as derivadas parciais:

\begin{equation}
\frac{\partial F}{\partial w_1} = (1 + \text{sigmoide}(-(w_1 \cdot \text{cor} + w_2 \cdot \text{diameter} + b))) \rightarrow \text{cor} \cdot \text{sigmoide}(-(w_1 \cdot \text{cor} + w_2 \cdot \text{diametro} + b))
\end{equation}

\begin{equation}
\frac{\partial F}{\partial w_2} = (1 + \text{sigmoide}(-(w_1 \cdot \text{cor} + w_2 \cdot \text{diâmetro} + b))) \rightarrow \text{diâmetro} \cdot \text{sigmoide} (-(w_1 \cdot \text{cor} + w_2 \cdot \text{diâmetro} + b))
\end{equation}

\begin{equation}
\frac{\partial F}{\partial b} = (1 + \exp(-(w_1 \cdot \text{cor} + w_2 \cdot \text{diâmetro} + b))) \rightarrow \text{sigmoide} (-(w_1 \cdot \text{cor} + w_2 \cdot \text{diâmetro} + b))
\end{equation}

\begin{equation}
\text{GradF}(w_1, w_2, b) = \left[ \begin{array}{c}
\text{cor} \cdot \text{sigmoide}(-(w_1 \cdot \text{cor} + w_2 \cdot \text{diâmetro} + b)), \\
\text{diâmetro} \cdot \text{sigmoide}(-(w_1 \cdot \text{cor} + w_2 \cdot \text{diâmetro} + b)), \\
\text{sigmoide}(-(w_1 \cdot \text{cor} + w_2 \cdot \text{diâmetro} + b))
\end{array} \right]
\end{equation}

\section{Resultados}

\begin{figure}[H]
  \centering
  \includegraphics[width=1\textwidth]{resultado.png}
  \caption{Output da classificação das flores para 1500 épocas}
\end{figure}

Como mostrado na figura acima, a classificação das rosas se aproxima de 0 e a classificação dos girassóis se aproxima de 1, isso demonstra a função sigmóide sendo executada, pois o intervalo de saída da classificação se encontra no intervalo [0,1], ou seja, caso a saída seja menor que 0.5, podemos considerar que a flor é uma rosa, e maior que 0.5 no caso dos girassóis.\\

\begin{figure}[H]
  \centering
  \includegraphics[width=1\textwidth]{resultado2.png}
  \caption{Output da classificação das flores para 500 épocas}
\end{figure}

Nesse caso, percebe-se que a precisão diminuiu devido a diminuição no número de épocas, ficando mais perto do limite da divisão de classificação.

\begin{figure}[H]
  \centering
  \includegraphics[width=0.8\textwidth]{grafico_classificacao.png}
  \caption{Gráfico da classificação de flores}
\end{figure}

Esse gráfico demonstra o limite entre a classificação da rosa e do girassol, onde o limite se estabelece por volta do ponto 0.5 no eixo y (cor), que é considerado a principal característica para realizar a classificação.

\begin{figure}[H]
  \centering
  \includegraphics[width=0.7\textwidth]{grafico_100.png}
  \caption{Gráfico do gradiente descendente para 100 épocas}
\end{figure}

\begin{figure}[H]
  \centering
  \includegraphics[width=0.7\textwidth]{grafico_500.png}
  \caption{Gráfico do gradiente descendente para 500 épocas}
\end{figure}

\begin{figure}[H]
  \centering
  \includegraphics[width=0.7\textwidth]{grafico_1500.png}
  \caption{Gráfico do gradiente descendente para 1500 épocas}
\end{figure}

Os três gráficos demonstram o gradiente descendente para três valores de épocas diferentes (100, 500 e 1500), mostrando o erro médio ao fim da quantidade de épocas. O valor de 1500 épocas apresenta uma margem de erro muito próxima de zero, mostrando um comportamento similar a uma reta ao fim das 1500 épocas, indicando estabilidade.

\section{Links úteis}

GitHub com todos os arquivos - https://github.com/pedrohlevy/redeneural\\
LinkedIn Diego Brunetto da Silva - https://www.linkedin.com/in/diego-brunetto-b892b1202/\\
LinkedIn João Pedro Borges Baeta - https://www.linkedin.com/in/joaopedrobaeta/\\
LinkedIn Pedro Henrique Levy Fermino Ferreira - https://www.linkedin.com/in/pedrohlevy/\\

\end{document}
